% Options for packages loaded elsewhere
\PassOptionsToPackage{unicode}{hyperref}
\PassOptionsToPackage{hyphens}{url}
\PassOptionsToPackage{dvipsnames,svgnames,x11names}{xcolor}
%
\documentclass[
]{article}
\usepackage{amsmath,amssymb}
\usepackage{lmodern}
\usepackage{iftex}
\ifPDFTeX
  \usepackage[T1]{fontenc}
  \usepackage[utf8]{inputenc}
  \usepackage{textcomp} % provide euro and other symbols
\else % if luatex or xetex
  \usepackage{unicode-math}
  \defaultfontfeatures{Scale=MatchLowercase}
  \defaultfontfeatures[\rmfamily]{Ligatures=TeX,Scale=1}
\fi
% Use upquote if available, for straight quotes in verbatim environments
\IfFileExists{upquote.sty}{\usepackage{upquote}}{}
\IfFileExists{microtype.sty}{% use microtype if available
  \usepackage[]{microtype}
  \UseMicrotypeSet[protrusion]{basicmath} % disable protrusion for tt fonts
}{}
\makeatletter
\@ifundefined{KOMAClassName}{% if non-KOMA class
  \IfFileExists{parskip.sty}{%
    \usepackage{parskip}
  }{% else
    \setlength{\parindent}{0pt}
    \setlength{\parskip}{6pt plus 2pt minus 1pt}}
}{% if KOMA class
  \KOMAoptions{parskip=half}}
\makeatother
\usepackage{xcolor}
\usepackage[margin=1in]{geometry}
\usepackage{color}
\usepackage{fancyvrb}
\newcommand{\VerbBar}{|}
\newcommand{\VERB}{\Verb[commandchars=\\\{\}]}
\DefineVerbatimEnvironment{Highlighting}{Verbatim}{commandchars=\\\{\}}
% Add ',fontsize=\small' for more characters per line
\usepackage{framed}
\definecolor{shadecolor}{RGB}{248,248,248}
\newenvironment{Shaded}{\begin{snugshade}}{\end{snugshade}}
\newcommand{\AlertTok}[1]{\textcolor[rgb]{0.94,0.16,0.16}{#1}}
\newcommand{\AnnotationTok}[1]{\textcolor[rgb]{0.56,0.35,0.01}{\textbf{\textit{#1}}}}
\newcommand{\AttributeTok}[1]{\textcolor[rgb]{0.77,0.63,0.00}{#1}}
\newcommand{\BaseNTok}[1]{\textcolor[rgb]{0.00,0.00,0.81}{#1}}
\newcommand{\BuiltInTok}[1]{#1}
\newcommand{\CharTok}[1]{\textcolor[rgb]{0.31,0.60,0.02}{#1}}
\newcommand{\CommentTok}[1]{\textcolor[rgb]{0.56,0.35,0.01}{\textit{#1}}}
\newcommand{\CommentVarTok}[1]{\textcolor[rgb]{0.56,0.35,0.01}{\textbf{\textit{#1}}}}
\newcommand{\ConstantTok}[1]{\textcolor[rgb]{0.00,0.00,0.00}{#1}}
\newcommand{\ControlFlowTok}[1]{\textcolor[rgb]{0.13,0.29,0.53}{\textbf{#1}}}
\newcommand{\DataTypeTok}[1]{\textcolor[rgb]{0.13,0.29,0.53}{#1}}
\newcommand{\DecValTok}[1]{\textcolor[rgb]{0.00,0.00,0.81}{#1}}
\newcommand{\DocumentationTok}[1]{\textcolor[rgb]{0.56,0.35,0.01}{\textbf{\textit{#1}}}}
\newcommand{\ErrorTok}[1]{\textcolor[rgb]{0.64,0.00,0.00}{\textbf{#1}}}
\newcommand{\ExtensionTok}[1]{#1}
\newcommand{\FloatTok}[1]{\textcolor[rgb]{0.00,0.00,0.81}{#1}}
\newcommand{\FunctionTok}[1]{\textcolor[rgb]{0.00,0.00,0.00}{#1}}
\newcommand{\ImportTok}[1]{#1}
\newcommand{\InformationTok}[1]{\textcolor[rgb]{0.56,0.35,0.01}{\textbf{\textit{#1}}}}
\newcommand{\KeywordTok}[1]{\textcolor[rgb]{0.13,0.29,0.53}{\textbf{#1}}}
\newcommand{\NormalTok}[1]{#1}
\newcommand{\OperatorTok}[1]{\textcolor[rgb]{0.81,0.36,0.00}{\textbf{#1}}}
\newcommand{\OtherTok}[1]{\textcolor[rgb]{0.56,0.35,0.01}{#1}}
\newcommand{\PreprocessorTok}[1]{\textcolor[rgb]{0.56,0.35,0.01}{\textit{#1}}}
\newcommand{\RegionMarkerTok}[1]{#1}
\newcommand{\SpecialCharTok}[1]{\textcolor[rgb]{0.00,0.00,0.00}{#1}}
\newcommand{\SpecialStringTok}[1]{\textcolor[rgb]{0.31,0.60,0.02}{#1}}
\newcommand{\StringTok}[1]{\textcolor[rgb]{0.31,0.60,0.02}{#1}}
\newcommand{\VariableTok}[1]{\textcolor[rgb]{0.00,0.00,0.00}{#1}}
\newcommand{\VerbatimStringTok}[1]{\textcolor[rgb]{0.31,0.60,0.02}{#1}}
\newcommand{\WarningTok}[1]{\textcolor[rgb]{0.56,0.35,0.01}{\textbf{\textit{#1}}}}
\usepackage{graphicx}
\makeatletter
\def\maxwidth{\ifdim\Gin@nat@width>\linewidth\linewidth\else\Gin@nat@width\fi}
\def\maxheight{\ifdim\Gin@nat@height>\textheight\textheight\else\Gin@nat@height\fi}
\makeatother
% Scale images if necessary, so that they will not overflow the page
% margins by default, and it is still possible to overwrite the defaults
% using explicit options in \includegraphics[width, height, ...]{}
\setkeys{Gin}{width=\maxwidth,height=\maxheight,keepaspectratio}
% Set default figure placement to htbp
\makeatletter
\def\fps@figure{htbp}
\makeatother
\setlength{\emergencystretch}{3em} % prevent overfull lines
\providecommand{\tightlist}{%
  \setlength{\itemsep}{0pt}\setlength{\parskip}{0pt}}
\setcounter{secnumdepth}{-\maxdimen} % remove section numbering
\ifLuaTeX
  \usepackage{selnolig}  % disable illegal ligatures
\fi
\IfFileExists{bookmark.sty}{\usepackage{bookmark}}{\usepackage{hyperref}}
\IfFileExists{xurl.sty}{\usepackage{xurl}}{} % add URL line breaks if available
\urlstyle{same} % disable monospaced font for URLs
\hypersetup{
  pdftitle={TP Statistiques 3},
  pdfauthor={Matthias LAPU , Amael KREIS},
  colorlinks=true,
  linkcolor={red},
  filecolor={Maroon},
  citecolor={Blue},
  urlcolor={blue},
  pdfcreator={LaTeX via pandoc}}

\title{TP Statistiques 3}
\author{Matthias LAPU , Amael KREIS}
\date{31 mars 2023}

\begin{document}
\maketitle

\hypertarget{estimation-du-maximum-de-vraisemblance-et-lintervalle-de-confiance}{%
\subsection{Estimation du Maximum de vraisemblance et L'intervalle de
confiance}\label{estimation-du-maximum-de-vraisemblance-et-lintervalle-de-confiance}}

\hypertarget{vraisemblance-la-loi-bernoulli}{%
\subsubsection{Vraisemblance: La loi
Bernoulli}\label{vraisemblance-la-loi-bernoulli}}

Soit \(X\) une variable aléatoire de Bernoulli (\texttt{rbinom}) avec
\(p=0.6\).

\begin{enumerate}
\def\labelenumi{\arabic{enumi}.}
\tightlist
\item
  Simuler un échantillon i.i.d de taille \(n=10\). Quelle est une façon
  simple d'estimer \(p\)?
\item
  Générer une fonction de vraisemblance, nommée \texttt{L\_bern}, en
  fonction de \((p, x)\), qui donne la vraisemblance d'un échantillon
  \(x=(x_1,\ldots,x_n)\) pour une valeur donnée de \(p\).
\item
  Pour votre échantillon, estimer la vraisemblance de l'échantillon pour
  \(n\) lois Bernoulli de paramètres \(p\) allant de 0 à 1. Tracez la
  courbe des valeurs calculées. Que remarquez-vous?
\item
  En utilisant la fonction \texttt{optim} de R, trouvez la valeur de
  \(p\) la plus probable d'avoir généré cet échantillon.
\end{enumerate}

Attention : \texttt{optim} est par défaut une routine de
\textbf{minimization}. Remarque : Avec la méthode de \emph{L-BFGS-B}
dans la fonction \texttt{optim}, vous pouvez traiter des contraintes sur
le(s) paramètre(s), lorsque c'est nécessaire.

\begin{enumerate}
\def\labelenumi{\arabic{enumi}.}
\setcounter{enumi}{4}
\item
  Tester avec des échantillons de taille allant de \(n=10\) à \(n=2000\)
  et comparer l'écart entre la valeur théorique attendue et la valeur
  obtenue. Que remarquez-vous? Comment combattre l'instabilité numérique
  due aux multiplications de probabilités?
\item
  Trouver deux intervalles de confiance de niveau 0.90 pour le paramètre
  \(p\), d'après (i) l'inégalité de Bienaymé-Chebycheff et (ii)
  l'inégalité de Hoeffding et les comparer. Incluent-ils la valeur
  réelle ?
\end{enumerate}

\hypertarget{vraisemblance-pour-plusieurs-paramuxe8tres-la-loi-gamma}{%
\subsubsection{Vraisemblance pour plusieurs paramètres: La Loi
Gamma}\label{vraisemblance-pour-plusieurs-paramuxe8tres-la-loi-gamma}}

Soit \(X_1,\ldots,X_n\) un échantillon de \(n\) variables indépendantes
de loi de Gamma(\(\alpha, \beta\)) où \(\theta = (\alpha, \beta)\) est
inconnue. Simuler un échantillon i.i.d de taille \(n=25\) avec
\(\theta_0 = (2.5, 1.5)\).

\begin{enumerate}
\def\labelenumi{\arabic{enumi}.}
\setcounter{enumi}{6}
\tightlist
\item
  Présentez l'histogramme des données simulées. Choisir trois paramètres
  candidats, disons, \(\theta_0\) (vrai) \(\theta_1, \theta_2\).
  Comparer l'histogramme avec les densités candidates. Que
  remarquez-vous?
\end{enumerate}

\begin{Shaded}
\begin{Highlighting}[]
\NormalTok{n }\OtherTok{=} \DecValTok{25}
\NormalTok{gamma0 }\OtherTok{\textless{}{-}} \FunctionTok{rgamma}\NormalTok{(n,}\FunctionTok{c}\NormalTok{(}\FloatTok{2.5}\NormalTok{,}\FloatTok{1.5}\NormalTok{))}
\FunctionTok{hist}\NormalTok{(gamma0)}
\end{Highlighting}
\end{Shaded}

\includegraphics{TP3_stat_2023_files/figure-latex/unnamed-chunk-1-1.pdf}

\begin{Shaded}
\begin{Highlighting}[]
\FunctionTok{hist}\NormalTok{(}\FunctionTok{rgamma}\NormalTok{(n,}\FunctionTok{c}\NormalTok{(}\DecValTok{3}\NormalTok{,}\DecValTok{4}\NormalTok{)))}
\end{Highlighting}
\end{Shaded}

\includegraphics{TP3_stat_2023_files/figure-latex/unnamed-chunk-1-2.pdf}

\begin{Shaded}
\begin{Highlighting}[]
\FunctionTok{hist}\NormalTok{(}\FunctionTok{rgamma}\NormalTok{(n,}\FunctionTok{c}\NormalTok{(}\DecValTok{2}\NormalTok{,}\DecValTok{6}\NormalTok{)))}
\end{Highlighting}
\end{Shaded}

\includegraphics{TP3_stat_2023_files/figure-latex/unnamed-chunk-1-3.pdf}

\begin{enumerate}
\def\labelenumi{\arabic{enumi}.}
\setcounter{enumi}{7}
\tightlist
\item
  Ecrire la log vraisemblance \texttt{logL\_gamma}. Générez une fonction
  de log-vraisemblance avec les arguments \((\theta, x)\), qui donne la
  log vraisemblance d'un échantillon pour une valeur donnée de
  \(\theta=(\alpha,\beta)\) et les donnéé \(x = (x_1,\ldots,x_n)\). Pour
  votre échantillon, estimer la log-vraisemblance de paramètre
  \(\theta=(\alpha,\beta)\), en faissant varier un paramètre à la fois.
\end{enumerate}

La formule de la vraisemblance de la loi gamma est : \[
L(\alpha ,\beta |x)={\frac {\beta ^{\alpha }}{\Gamma (\alpha )}}x^{\alpha -1}\exp(-\beta x)
\] \[
\log L(\alpha ,\beta |x)=\alpha \log \beta -\log \Gamma (\alpha )+(\alpha -1)\log x-\beta x
\]

\begin{Shaded}
\begin{Highlighting}[]
\NormalTok{logL\_gamma\_unit }\OtherTok{\textless{}{-}} \ControlFlowTok{function}\NormalTok{(alpha,beta,x)\{}
  \FunctionTok{return}\NormalTok{( alpha}\SpecialCharTok{*}\FunctionTok{log}\NormalTok{(beta) }\SpecialCharTok{{-}} \FunctionTok{log}\NormalTok{(}\FunctionTok{gamma}\NormalTok{(alpha)) }\SpecialCharTok{+}\NormalTok{ (alpha}\DecValTok{{-}1}\NormalTok{)}\SpecialCharTok{*}\FunctionTok{log}\NormalTok{(x) }\SpecialCharTok{{-}}\NormalTok{ beta}\SpecialCharTok{*}\NormalTok{x ) }
\NormalTok{\}}

\CommentTok{\#on fait varier un paramètre et on calcule la somme }
\NormalTok{logL\_gamma }\OtherTok{\textless{}{-}} \ControlFlowTok{function}\NormalTok{(alpha,beta,interval)\{}
\NormalTok{  sum }\OtherTok{\textless{}{-}} \DecValTok{0}
  \ControlFlowTok{for}\NormalTok{(i }\ControlFlowTok{in}\NormalTok{ interval) \{}
\NormalTok{    sum }\OtherTok{=}\NormalTok{  sum }\SpecialCharTok{+} \FunctionTok{logL\_gamma\_unit}\NormalTok{(alpha,beta,i)}
\NormalTok{  \}}
  \FunctionTok{return}\NormalTok{ (sum)}
\NormalTok{\}}
\end{Highlighting}
\end{Shaded}

\begin{Shaded}
\begin{Highlighting}[]
\NormalTok{array }\OtherTok{\textless{}{-}} \FunctionTok{c}\NormalTok{()}
\NormalTok{interval }\OtherTok{\textless{}{-}} \FunctionTok{seq}\NormalTok{(}\DecValTok{1}\NormalTok{,}\DecValTok{10}\NormalTok{)}
\ControlFlowTok{for}\NormalTok{(i }\ControlFlowTok{in}\NormalTok{ interval)\{}
  \ControlFlowTok{for}\NormalTok{(j }\ControlFlowTok{in}\NormalTok{ interval)\{}
\NormalTok{    array }\OtherTok{\textless{}{-}} \FunctionTok{c}\NormalTok{(array,}\FunctionTok{logL\_gamma}\NormalTok{(i,j,gamma0));}
\NormalTok{  \}}
\NormalTok{\}}
\FunctionTok{print}\NormalTok{(}\FunctionTok{mean}\NormalTok{(array)); }\CommentTok{\# la moyenne d\textquotesingle{}un tableau}
\end{Highlighting}
\end{Shaded}

\begin{verbatim}
## [1] -124.6043
\end{verbatim}

\begin{enumerate}
\def\labelenumi{\arabic{enumi}.}
\setcounter{enumi}{8}
\item
  Tracer la courbe des valeurs ainsi calculées. Comme la fonction,
  \(\ell(\theta)\), a deux arguments, vous pouvez réaliser des tracés de
  contour (\texttt{contour}). Pour plus de simplicité, il suffit de
  tracer comme une fonction unidimensionnelle en supposant que l'autre
  est fixe: \(\ell(\alpha|\beta=\beta_0)\) avec quelque \(\beta_0\) de
  votre choix et \(\ell(\beta| \alpha=\alpha_0)\) pour quelque
  \(\alpha_0\) de votre choix. Que remarquez-vous?
\item
  Donner l'expression mathématique du vecteur Score (les dérivées
  premières) à laquelle l'EMV répond. En utilisant la fonction
  \texttt{optim}, trouver la valeur de \(\theta\) la plus probable pour
  votre l'échantillon.
\end{enumerate}

Le score de la fonction de vraisemblance est le vecteur des dérivées
premières de la fonction de log-vraisemblance par rapport au paramètre
θ.

Score(θ) = ∇\_θ log L(θ)

où ∇\_θ représente le gradient par rapport à θ et log L(θ) est le
logarithme de la fonction de vraisemblance.

\begin{enumerate}
\def\labelenumi{\arabic{enumi}.}
\setcounter{enumi}{10}
\item
  Répéter l'estimation 100 fois avec de nouveaux ensembles de données
  (échantillons) et tracer les estimations \(\hat\alpha\) vs
  \(\hat\beta\). Sont-elles indépendantes ? Trouver l'intervalle où 95\%
  des estimations sont incluses pour chaque paramètre. Vous venez de
  trouver un intervalle de confiance empirique à 95\% ! Visualisez vos
  résultats à l'aide d'un histogramme. Que remarquez-vous?
\item
  Tester avec des échantillons de taille \(n=15\) et \(n=100\) et
  comparer avec les résultats précédents. Quel est l'effet de la taille
  de l'échantillon ?
\end{enumerate}

\hypertarget{normalituxe9-asymptotique-de-lemv-et-lintervalle-de-confidence}{%
\subsubsection{Normalité asymptotique de l'EMV et l'intervalle de
confidence}\label{normalituxe9-asymptotique-de-lemv-et-lintervalle-de-confidence}}

En pratique, nous n'avons qu'un seul ensemble de données avec des
paramètres \(\theta=(\theta_1,\ldots,\theta_p)\) inconnus, la simulation
de multiples échantillons permettant de construire la distribution de
l'estimateur n'est donc plus possible. Dans ce cas, on peut construire
des intervalles basés sur une approximation asymptotique de la
distribution, en utilisant le fait que
\(\hat{\theta}\approx \mathcal{N}(\theta,I_n(\theta)^{-1})\).

L'évaluation de l'information de Fisher directe nécessite de calculer le
hessien \(H\) et son espérance analytiquement en différenciant l'opposé
de la log-vraisemblance deux fois par rapport aux paramètres (pour
obtenir la matrice complète), puis d'inverser explicitement la matrice,
et enfin de remplacer \(\theta\) par \(\hat{\theta}\).

\begin{enumerate}
\def\labelenumi{\arabic{enumi}.}
\setcounter{enumi}{12}
\tightlist
\item
  Pour revenir à l'exemple du Gamma (\(n=25\)), trouver l'information de
  Fisher et estimer la covariance asymptotique. A partir de là,
  construire un intervalle de confiance asymptotique de niveau \(0.95\).
  Comparer avec la solution de 11.
\end{enumerate}

En général, il n'est pas possible d'évaluer l'information de Fisher de
manière analytique, ou cela peut prendre trop de temps. Dans ce cas, on
utilise l' ``information observée'' (sans ésperance),
\(I_O(\hat{\theta})^{-1}= -H(\hat{\theta})^{-1}\).

Nous pouvons utiliser l'option `hessian=TRUE' dans \texttt{optim} pour
obtenir la matrice hessienne et estimer la fonction de covariance par
son inverse.

\begin{enumerate}
\def\labelenumi{\arabic{enumi}.}
\setcounter{enumi}{13}
\item
  Estimer la covariance asymptotique avec l'information observée et
  construire un intervalle de confiance asymptotique de niveau \(0.95\).
\item
  Nous utilisons la simulation pour comparer la performance de ces deux
  estimateurs. Simuler plusieurs fois un nouvel ensemble de données,
  construiser les deux intervalles de confiance et compter combien de
  fois cet intervalle contient les vraies valeurs. Quelle est votre
  conclusion?
\end{enumerate}

\end{document}
